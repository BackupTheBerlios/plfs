\chapter{Comunicaci�n entre los componentes}

\textbf{\textit{En este cap�tulo comentamos diferentes implementaciones de la
comunicaci�n entre componentes del sistema. Como veremos, nos centraremos en
los tipos de mensajes en cada tipo de comunicaci�n y la seguridad que debemos
implementar en cada una, para asegurar autenticidad y confidencialidad.}}
\\

Las comunicaciones entre componentes de un mismo nodo funcionan a modo de
llamadas ``normales'' entre partes de un mismo programa (aunque se carguen en
forma de \textit{plugins}).

La comunicaci�n entre componentes de nodos separados se realiza siempre mediante
el paso de mensajes que siguen �ste formato:

\begin{enumerate}
	\item IDoperaci�n

	\item Remitente

	\item Datos de la operaci�n (par�metros o resultado)
\end{enumerate}

Seguidamente pasamos a concretar los detalles de comunicaci�n de los diferentes
componentes.



\section{\kplfs $\longleftrightarrow$ \uplfs}

Para que la aplicaci�n en modo usuario pueda enterarse de las operaciones
que se realizan sobre el sistema de ficheros, el m�dulo \kplfs a nivel de
kernel, debe comunicarse con el m�dulo \uplfs a nivel usuario.
Para realizar esta comunicaci�n de eventos, hay varias posibilidades:

\begin{description}
	\item [Dispositivo:]
		Aprovecha las propias caracter�sticas de un dispositivo de
		sistema, ya que permite hacer esperas no activas (es decir
		bloqueantes), a trav�s de \textit{poll}, \textit{read},
		\textit{select} de un dispositivo que implementa \kplfs.

	\item [Compartici�n de memoria:]
		El problema de esta soluci�n es que requiere una espera activa
		por parte de \uplfs, que debe ir comprobando la zona de memoria
		compartida para ver si hay alg�n nuevo evento procedente de
		\kplfs.
\end{description}

As� pues, por las ventajas de operaciones no bloqueantes que ofrece,
utilizaremos un \textbf{dispositivo} para comunicar ambos componentes.

Cuando se realice una operaci�n (\texttt{ls},\texttt{mv}, etc) en el sistema
de ficheros \plfs, el m�dulo de kernel \kplfs que implementa estas operaciones,
deber� informar dichos eventos al m�dulo a nivel usuario \uplfs (el cual deber�
realizar las operaciones pertinentes, comunic�ndose con los diferentes
componentes disponibles y respondiendo a cada operaci�n con su identificador
\texttt{id}).

En �ste caso, aunque los componentes est�n en el mismo nodo, deben enviarse
mensajes a trav�s del dispositivo de control, mensajes que tienen el mismo
formato que se ha comentado al inicio del cap�tulo, pero sin el par�metro de
\texttt{Remitente}.
\\

\kplfs $\rightarrow$ \uplfs:

\begin{itemize}
	\item \texttt{execute (component, id, operation, \ldots)}
\end{itemize}

\kplfs $\leftarrow$ \uplfs:

\begin{itemize}
	\item Operaciones para el retorno del resultado de las peticiones del
		VFS a \uplfs (ver \cite{lkapi}).
	\item \texttt{invalidate (slice\_name, node\_name, type, path)}
\end{itemize}



\section{\uplfs/\dpld/\dplc $\rightarrow$ \pldb}

Llamada interna de funciones.
\\

\uplfs/\dpld/\dplc $\rightarrow$ \pldb:

\begin{itemize}
	\item \texttt{getSlices ()}
	\item \texttt{getNodes ()}
	\item \texttt{getSliceNodes (slice\_name)}
	\item \texttt{getSliceNodesAny (slice\_name, num)}
	\item \texttt{getSliceNodesNumber (slice\_name)}
	\item \texttt{getSliceNode (slice\_name, node\_name)}
	\item \texttt{getSliceNodeAny (slice\_name)}
	\item \texttt{getSliceNodeNearest (slice\_name)}
	\item \texttt{getNodeSlices (node\_name)}
\end{itemize}



\section{\uplfs $\longleftrightarrow$ \dpld}

Llamada interna de funciones.
\\

\uplfs $\rightarrow$ \dpld:

\begin{itemize}
	\item \texttt{deployToSlice (slice\_name, path\_from, path\_to)}
	\item \texttt{deployToNode (slice\_name, node\_name, path\_from,
		path\_to)}
	\item \texttt{getInfoShared (slice\_name, path)}
	\item \texttt{getInfoUnshared (slice\_name, node\_name, path)}
	\item \texttt{getFileShared (slice\_name, path)}
	\item \texttt{getFileUnshared (slice\_name, node\_name, path)}
	\item \texttt{deleteFileShared (slice\_name, path)}
	\item \texttt{deleteFileUnshared (slice\_name, node\_name, path)}
	\item \texttt{getSliceNodeVersion (slice\_name, node\_name)}
	\item \texttt{registerNeeded (slice\_name, node\_name)}
\end{itemize}

\uplfs $\leftarrow$ \dpld:

\begin{itemize}
	\item \texttt{invalidate (slice\_name, node\_name, type, path)}
\end{itemize}



\section{\dpld $\longleftrightarrow$ \dplc}
\label{sect:warnings}

Teniendo en cuanta que cada componente tiene su propio par de llaves
(un par $K_{pub_{d}}$ y $K_{priv_{d}}$ para cada \dpld y un mismo par
$K_{pub_{c}}$ y $K_{priv_{c}}$ para todos los \dplc), todas las comunicaciones
van cifradas con la clave p�blica del emisor (para asegurar que s�lo
los receptores puedan leer los contenidos) y firmadas con la clave
privada del emisor (para poder as� comprobar su autenticidad), excepto
\texttt{getKey}, que s�lo va firmada.

Luego, en el destino, se comprueba la firma, se descifran los datos y se
procede a realizar la operaci�n pertinente.
\\

\dpld $\rightarrow$ \dplc:

\begin{itemize}
	\item \texttt{deployToSlice (slice\_name, path\_to, file, type,
		new\_version)}
	\item \texttt{getInfo (slice\_name, path)}
	\item \texttt{getFile (slice\_name, path)}
	\item \texttt{deleteFile (slice\_name, path, new\_version)}
	\item \texttt{getVersion (slice\_name)}
	\item \texttt{getKey ($K_{pub_{d}}$)}
	\item \texttt{registerWithResponse (slice\_name, $K_{SSH}$, $P_{SSH}$,
		$K_{pub}$)}
	\item \texttt{register (slice\_name, $K_{SSH}$, $P_{SSH}$, $K_{pub}$)}
\end{itemize}

\dpld $\leftarrow$ \dplc:

\begin{itemize}
	\item \texttt{registerNeeded (slice\_name, node\_name)}
	\item \texttt{invalidate (slice\_name, node\_name, type, path)}
\end{itemize}



\section{\dplc $\rightarrow$ \dplc}

Como la red de \dplc intenta llegar sola a la coherencia de versiones, necesita
ofrecer m�todos para ponerse en contacto e intercambiar datos, cifrando y
firmando con $K_{priv_{c}}$.

\begin{itemize}
	\item \texttt{deployToSlice (slice\_name, path\_to, file, type,
		new\_version)}
	\item \texttt{needDeploy (slice\_name, node\_name, path)}
	\item \texttt{ping (slice\_name, node\_name, version)}
	\item \texttt{getSSH (slice\_name)}
\end{itemize}



\section{\dpld/\dplc $\longleftrightarrow$ \umcc}

Llamada interna de funciones.

Cuando tanto \dpld como \dplc quieren comunicarse con un grupo (\dpld se
comunica con el grupo de receptores del slice para mandarles operaciones y
\dplc se comunica con el grupo de emisores para mandarles avisos), lo hacen a
trav�s de la red multicast, mediante el componente de entrada a ella que es
\umcc, de forma que mandan los datos como si fueran directamente como la
comunicaci�n anterior y se transmiten transparentemente por la red \umc.
\\

\dpld/\dplc $\rightarrow$ \umcc:

\begin{itemize}
	\item \texttt{group\_id getGroup (group\_name)}
	\item \texttt{setOptions (group\_id, options[])}
	\item \texttt{getOptions (group\_id, options[])}
	\item \texttt{sendData (group\_id, data)}
	\item \texttt{recieveData (group\_id, data)}
	\item \texttt{joinSender (group\_id, node\_name)}
	\item \texttt{deleteSender (group\_id, node\_name)}
	\item \texttt{joinReceiver (group\_id, node\_name)}
	\item \texttt{deleteReceiver (group\_id, node\_name)}
\end{itemize}

Con tal de saber \dpld cu�l es el estado de los slices, como la red multicast
subyacente puede informar de la adici�n y eliminaci�n de receptores, es el
propio \umcc quien se encarga de avisar a \dpld (en lugar de \dplc, evitando as�
ocupar la red innecesariamente).
\\

\dpld $\leftarrow$ \umcc:

\begin{itemize}
	\item \texttt{addNodeToSlice (slice\_name, node\_name)}
	\item \texttt{removeNodeFromSlice (slice\_name, node\_name)}
\end{itemize}

\dplc $\leftarrow$ \umcc son:

\begin{itemize}
	\item \texttt{joined (slice\_name)}
	\item \texttt{deleted (slice\_name)}
\end{itemize}



\section{\dplc $\longleftrightarrow$ \famon}

Llamada interna de funciones.
\\

\dplc $\rightarrow$ \famon:

\begin{itemize}
	\item \texttt{watch (path)}
	\item \texttt{unwatch (path)}
	\item \texttt{stop ()}
\end{itemize}

\dplc $\leftarrow$ \famon:

\begin{itemize}
	\item \texttt{invalidate (slice\_name, path)}
\end{itemize}
