\chapter{Comunicaci�n entre los componentes}

%TODO: se�alar qu� comunicaciones van encriptadas/desencriptadas

\textbf{\textit{En este cap�tulo comentamos diferentes implementaciones de la
comunicaci�n entre componentes del sistema. Como veremos, nos centraremos en
los tipos de mensajes en cada tipo de comunicaci�n y la seguridad que debemos
implementar en cada una, para asegurar autenticidad y confidencialidad.}}

Las comunicaciones entre componentes de un mismo nodo funcionan a modo de
llamadas ``normales'' entre partes de un mismo programa (aunque se carguen en
forma de \textit{plugins}).

La comunicaci�n entre componentes de nodos separados se realiza siempre mediante
el paso de mensajes que siguen �ste formato:

\begin{enumerate}
	\item IDoperaci�n

	\item Remitente

	\item Datos de la operaci�n (par�metros o resultado)
\end{enumerate}

Seguidamente pasamos a concretar los detalles de comunicaci�n de los diferentes
componentes.



\section{\kplfs $\longleftrightarrow$ \uplfs}

Para que la aplicaci�n en modo usuario pueda enterarse de las operaciones
que se realizan sobre el sistema de ficheros, el m�dulo \kplfs a nivel de
kernel, debe comunicarse con el m�dulo \uplfs a nivel usuario.
Para realizar esta comunicaci�n de eventos, hay varias posibilidades:

\begin{description}
	\item [Dispositivo:]
		Aprovecha las propias caracter�sticas de un dispositivo de sistema,
		ya que permite hacer esperas no activas (es decir bloqueantes), a
		trav�s de \textit{poll}, \textit{read}, \textit{select} de un
		dispositivo que implementa \kplfs.

	\item [Compartici�n de memoria:]
		El problema de esta soluci�n es que requiere una espera activa por
		parte de \uplfs, que debe ir comprobando la zona de memoria compartida
		para ver si hay alg�n nuevo evento procedente de \kplfs.
\end{description}

As� pues, por las ventajas de operaciones no bloqueantes que ofrece,
utilizaremos un \textbf{dispositivo} para comunicar ambos componentes.

Cuando se realice una operaci�n (\texttt{ls},\texttt{mv}, etc) en el sistema
de ficheros \plfs, el m�dulo de kernel \kplfs que implementa estas operaciones,
deber� informar dichos eventos al m�dulo a nivel usuario \uplfs (el cual deber�
realizar las operaciones pertinentes, comunic�ndose con los diferentes
componentes disponibles).

En �ste caso, aunque los componentes est�n en el mismo nodo, deben enviarse
mensajes a trav�s del dispositivo de control, mensajes que tienen el mismo
formato que se ha comentado al inicio del cap�tulo, pero sin el par�metro de
\texttt{Remitente}.



\section{\uplfs/\dpld $\rightarrow$ \pldb}

Llamada interna de funciones.



\section{\uplfs $\longleftrightarrow$ \dpld}

Llamada interna de funciones.



\section{\dpld $\rightarrow$ \dplc}

Todas las comunicaciones van cifradas con $K_{pub_{c}}$ (para asegurar que s�lo
los \dplc puedan leer los contenidos) y firmadas con $K_{priv_{d}}$ (para poder
as� comprovar su autenticidad).

Luego, en el destino, se comprueba la firma, se descifran los datos y se
procede a realizar la operaci�n.



\section{\dplc $\rightarrow$ \dpld}
\label{sect:warnings}

Todas las comunicaciones van cifradas con $K_{pub_{d}}$ (para asegurar que s�lo
los \dplc puedan leer los contenidos) y firmadas con $K_{priv_{c}}$ (para poder
as� comprovar su autenticidad).

Luego, en el destino, se comprueba la firma, se descifran los datos y se
procede a realizar la operaci�n.



\section{\dpld/\dplc $\rightarrow$ \umcc}

Cuando tanto \dpld como \dplc quieren comunicarse con un grupo (\dpld se
comunica con el grupo de receptores del slice para mandarles operaciones y
\dplc se comunica con el grupo de emisores para mandarles avisos), lo hacen a
trav�s de la red multicast, mediante el componente de entrada a ella que es
\umcc, de forma que mandan los datos como si fueran directamente una de las dos
comunicaciones anteriores y se transmiten transparentemente por la red \umc.
