\chapter{Qu� y porqu� lo queremos hacer?}

Tanto la documentaci�n como la implementaci�n se encuentran disponibles en la
p�gina del proyecto \cite{PLFS}, que basa su idea inicial en la propuesta de
\cite{ParPipelinedDist}.

Cabe notar que, como inicialmente nos centraremos en las operaciones de
despliegue, siempre que nos refiramos a \plfs, s�lo estaremos haciendo
referencia, realmente, al subconjunto dedicado al despliegue.

El proyecto pretende juntar b�sicamente dos conceptos con tal de conseguir dos
grandes objetivos:



\section{Facilidad de uso}

A trav�s de una interf�cie que todo el mundo ya conozca y necesite un m�nimo
de aprendizaje y permita un f�cil desarrollo de aplicaciones de m�s alto
nivel.

Este objetivo pretende conseguirse a trav�s de la implementaci�n de un sistema
de ficheros, con una estructura a�n por determinar, pero que permita desplegar
una aplicaci�n de una forma tan f�cil como copiar un fichero en un directorio
concreto, evento que disparar�a el sistema de distribuci�n de ficheros.



\section{Eficiencia en la transferencia de ficheros}

Como la aplicaci�n a desplegar necesita ser copiada a todos los nodos de un
slice, no basta con copiarla directamente a un sistema de ficheros "en red"
convencional, puesto que se enviar�a a un punto central y luego todos los
nodos, al arrancar el servicio, leer�an a la vez el programa para poder
ponerlo en marcha, saturando innecesariamente la red.

Para �ste fin, la idea de una distribuci�n multicast es ideal, ya que
permitir�a ahorrar al m�ximo el ancho de banda, permitiendo un despliegue
en paralelo y evitando al m�ximo las aglomeraciones de datos innecesarias.

Como la mayor�a de routers de Internet no soportan el enrutamiento multicast,
es por eso que har�a falta un servicio en modo usuario que se encargara de
hacer �sta distribuci�n.

